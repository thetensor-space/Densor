\documentclass{documentation}

\title{Densor Package}

\author{Joshua Maglione}
\address{Universit\"at Bielefeld}
\email{jmaglione@math.uni-bielefeld.de}

\author{James B. Wilson}
\address{Colorado State University}
\email{james.wilson@colostate.edu}

\version{1.0}
\date{\today}
\copyrightyear{2016--2019}

\input{preamble.tex}

%-----------------------------------------------------------------------------
\begin{document}

\frontmatter

\dominitoc
\maketitle
\tableofcontents

\mainmatter

\chapter{Introduction}




\chapter{Subspaces as closures}~

The subspaces in this section are given by Galois connections in \cite{FMW:densors}*{Theorem~A}. 
Currently, these are the only closures that are constructible in this package, and they are implemented for 3-tensors.

\index{DerivationClosure}
\begin{intrinsics}
DerivationClosure(T, Delta) : TenSpc, AlgMat -> TenSpc
DerivationClosure(T, Delta) : TenSpc, ModMatFld -> TenSpc
DerivationClosure(T, Delta) : TenSpc, AlgMatLie -> TenSpc
DerivationClosure(T, Delta) : TenSpc, [Mtrx] -> TenSpc
DerivationClosure(T, Delta) : TenSpc, [AlgMatLie] -> TenSpc
\end{intrinsics}

Returns the derivation closure of the given tensor space $T$, with frame $U_2\times U_1\rightarrowtail U_0$, with operators $\Delta\subseteq \text{End}(U_2)\times \text{End}(U_1)\times \text{End}(U_0)$.
If a nontrivial subset of coordinates are fused, then \texttt{DerivationClosure} accepts $\Delta$ input from the smaller ambient space. 
For example, if coordinates $2$ and $1$ are fused, then the input for $\Delta$ can be in either $\text{End}(U_2)\times\text{End}(U_1)\times\text{End}(U_0)$ or $\text{End}(U_2)\times\text{End}(U_0)$.
Currently, this only works for tensor spaces of valence 3.
This is the subspace whose tensors' derivation algebra contains $\Delta$.

\index{DerivationClosure}
\begin{intrinsics}
DerivationClosure(T, t) : TenSpc, TenSpcElt -> TenSpc
\end{intrinsics}

Returns the derivation closure of the given tensor space $T$, with frame $U_2\times U_1\rightarrowtail U_0$, whose operators are the derivation algebra of $t$. 
Currently, this only works for tensor spaces of valence 3.
This is the subspace whose tensors' derivation algebra contains the derivation algebra of $t$.

\index{UniversalDensorSubspace}
\begin{intrinsics}
UniversalDensorSubspace(t) : TenSpcElt -> TenSpc
\end{intrinsics}

Returns the universal densor subspace of the given tensor $t : U_2\times U_1 \rightarrowtail U_0$ as a subspace of the universal tensor space of $t$, whose operators are the derivation algebra of $t$. 
Currently, this only works for tensor spaces of valence 3.
This is equivalent to \texttt{DerivationClosure(Parent(t), t)}. 
This tensor subspace is stored with $t$ once computed.

\begin{example}[Rank1Densor]

We illustrate the fact that the tensor for (hyper-)matrix multiplication spans its densor, see \cite{FMW:densors}*{Theorem~G}.
We will construct the derivation closure of $\bra{t}:\mathbb{M}_{2\times 3}(\mathbb{F}_3)\times \mathbb{M}_{3\times 2}(\mathbb{F}_3)\rightarrowtail \mathbb{M}_{2}(\mathbb{F}_3)$, given by matrix multiplcation. 
The derivation closure of $t$ in the universal tensor space $T$ is 1-dimensional.
\begin{code}
> Fr := [ KMatrixSpace(GF(3),2,3), KMatrixSpace(GF(3),3,2),
>     KMatrixSpace(GF(3),2,2) ];
> F := func< x | x[1]*x[2] >;
> t := Tensor(Fr, F);
> t;
Tensor of valence 3, U2 x U1 >-> U0
U2 : Full Vector space of degree 6 over GF(3)
U1 : Full Vector space of degree 6 over GF(3)
U0 : Full Vector space of degree 4 over GF(3)
\end{code}

The derivation algebra of $t$ is isomorphic to $(\mathfrak{gl}_2(\mathbb{F}_3)\oplus \mathfrak{gl}_3(\mathbb{F}_3)\oplus \mathfrak{gl}_2(\mathbb{F}_3))/\mathbb{F}_3$.
We do not show this, but we verify that the dimensions match.
\begin{code}
> D := DerivationAlgebra(t);
> Dimension(D) eq 4+9+4-1;
true
\end{code}

Now we verify that $t$ spans its own densor.
\begin{code}
> T := Parent(t);
> T;
Tensor space of dimension 144 over GF(3) with valence 3
U2 : Full Vector space of degree 6 over GF(3)
U1 : Full Vector space of degree 6 over GF(3)
U0 : Full Vector space of degree 4 over GF(3)
> densor := DerivationClosure(T, D);
> densor eq sub< T | t >;
true
\end{code}
\end{example}


\index{NucleusClosure}
\begin{intrinsics}
NucleusClosure(T, Delta, a, b) : TenSpc, AlgMat, RngIntElt, RngIntElt -> TenSpc
NucleusClosure(T, Delta, a, b) : TenSpc, ModMatFld, RngIntElt, RngIntElt -> TenSpc
NucleusClosure(T, Delta, a, b) : TenSpc, [Mtrx], RngIntElt, RngIntElt -> TenSpc
\end{intrinsics}

Returns the nucleus closure of the tensor space $T$, with frame $U_2\times U_1\rightarrowtail U_0$, with operators $\Delta\subseteq \text{End}(U_a)\times \text{End}(U_b)$.
Currently, this only works for tensor spaces of valence 3.
This returns the subspace whose tensors' $\{a,b\}$-nuclues contains $\Delta$.

\index{NucleusClosure}
\begin{intrinsics}
NucleusClosure(T, t, a, b) : TenSpc, TenSpcElt, RngIntElt, RngIntElt -> TenSpc
\end{intrinsics}

Returns the nucleus closure of the tensor space $T$, with frame $U_2\times U_1\rightarrowtail U_0$, whose operators are the $\{a,b\}$-nucleus of $t$.
Currently, this only works for tensor spaces of valence 3.
This returns the subspace whose tensors' $\{a,b\}$-nuclues contains the $\{a,b\}$-nucleus of $t$.

\begin{example}[NucClosure]

We illustrate that if $t$ is the commutator tensor from the Heisenberg group, then the densor of $t$ is $\text{Cen}(t)\cdot t$, 1-dimensional over the centroid.
First, we construct $t$ with frame $\mathbb{F}_5^6\times\mathbb{F}_5^6\rightarrowtail \mathbb{F}_5^3$, so that $t$ is $\mathbb{F}_5$-bilinear.
\begin{code}
> H := ClassicalSylow(GL(3,125), 5);
> t := pCentralTensor(H);
> t;
Tensor of valence 3, U2 x U1 >-> U0
U2 : Full Vector space of degree 6 over GF(5)
U1 : Full Vector space of degree 6 over GF(5)
U0 : Full Vector space of degree 3 over GF(5)
\end{code}

Because $t$ came from the Heisenberg group over $\mathbb{F}_{5^3}$, the centroid of $t$ will be 3-dimensional (over $\mathbb{F}_5$) and isomorphic to $\mathbb{F}_{5^3}$.
After computing the centroid of $t$, we will rewrite $t$ over the centroid $\text{Cen}(t)$, so that $t$ is $\mathbb{F}_{5^3}$-bilinear.
\begin{code}
> C := Centroid(t);
> C;
Matrix Algebra of degree 15 and dimension 3 with 3 generators over GF(5)
> s := TensorOverCentroid(t);
> s;
Tensor of valence 3, U2 x U1 >-> U0
U2 : Full Vector space of degree 2 over GF(5^3)
U1 : Full Vector space of degree 2 over GF(5^3)
U0 : Full Vector space of degree 1 over GF(5^3)
\end{code}

Now we will compute the nucleus closure of both $s$ and $t$ at $\{2,1\}$.
The dimension of the closure of $s$ will be 1-dimensional, while the dimension of the closure for $t$ will not.
\begin{code}
> NucleusClosure(Parent(s), s, 2, 1);
Tensor space of dimension 1 over GF(5^3) with valence 3
U2 : Full Vector space of degree 2 over GF(5^3)
U1 : Full Vector space of degree 2 over GF(5^3)
U0 : Full Vector space of degree 1 over GF(5^3)
> 
> NucleusClosure(Parent(t), t, 2, 1);
Tensor space of dimension 9 over GF(5) with valence 3
U2 : Full Vector space of degree 6 over GF(5)
U1 : Full Vector space of degree 6 over GF(5)
U0 : Full Vector space of degree 3 over GF(5)
\end{code}
\end{example}



\backmatter

\begin{bibdiv}
\begin{biblist}

\bib{FMW:densors}{article}{
   author={First, Uriya},
   author={Maglione, Joshua},
   author={Wilson, James B.},
   title={Polynomial identity tensors and their invariants},
   note={in preparation},
}

\end{biblist}
\end{bibdiv}

\printindex


\end{document}
